\chapter{Definição dos requisitos, atores e casos de uso}

\section{Atores}
\begin{description}
    \item[Desenvolvedor] Desenvolvedores do sistema em si. Esse
    projeto esta em constante contrução e aperfeiçoamento, logo o
    desenvolvedor.

    \item[Administrador] É um analista com maiores permissões. Além de
    analisar, administra o projeto e os analistas.

    \item[Analista] Usuário principal do sistema, faz análises de
    conteúdo buscado na internet.

    \item[Cliente] Empresa ou pessoa que contratou o serviço
    oferecido.
\end{description}


\section{Requisitos}
\begin{itemize}
    \item A interface gráfica tem que ser acessível via um browser
    pela internet.
    \item Dado um conjunto de palavras chaves o sistema tem de ter a
    capacidade de coletar conteúdo em diversos canais(orkut, flickr,
            blog, twitter).
    \item O sistema deve ter uma interface gráfica que mostra o
    conteudo coletado.
    \item O sistema deve ter uma interface que dá capacidade de se
    analisar o conteúdo coletado.
    \item O sistema deve ser capaz de gerar gráficos sobre a análise
    feita.
\end{itemize}

\section{Requisitos de Dependabilidade}
\begin{description}
    \item[Disponibilidade do servidor] O sistema tem que ficar no ar
    durante pelo menos 90\% do tempo durante os horários de análise -
    8:00 as 20:00.
    \item[Safety] O sistema deve fazer um backup do banco de dados de
    12 em 12 horas.
    \item[Security] O sistema deve manter dados de usuários(login e
            senhas) protegidos.
    \item[Security] O sistema deve ter uma divisão de privilégios,
    sendo que algumas informações só podem ser vistas por certos
    usuários.
\end{description}
 


\chapter{Casos de Uso}

\begin{itemize}
    \item Usuário $->$ loga no sistema

Todo usuário, deve poder logar na ferramenta através de seu usuário e senha.



    \item Analista $->$ Loga no sistema $->$ Analise um projeto

O analista, entra no sistema e após escolher o projeto e o
canal(orkut, youtube...) faz a análise  de conteúdo e salva essa
análise no banco de dados.


\item Analista $->$ Loga no Sistema $->$ Visualiza Gráficos

O analista loga no sistema, escolhe um projeto e visualiza os gráficos do conteúdo.



\item Administrador $->$ Loga no Sistema $->$ Cria um projeto 

Administrador loga no sistema, e cria um projeto, com data inicial e final e palavras chaves.


\item Administrador $->$ Loga no Sistema $->$ Adiciona analista

Administrador loga no sistema, adiciona um analista e o associa a um projeto.



\item Adminstrador $->$ Loga no Sistema $->$ Cria relatório

Administrador loga no sistema, entra num projeto e cria um relatório e envia ao cliente.



\item Cliente $->$ Loga no sistema $->$ Visualiza informações provida pelo administrador sobre um determinado projeto

Cliente loga no sistema, visualiza relatório provido pelo administrador

\end{itemize}
