\section{Conclusão}

A realização deste trabalho se mostrou de suma importância, para o
aprofundamento nos conceitos de dependabilidade, pois permitiu a visualização
da ferramenta por diversos aspectos como: disponibilidade(Availability); confiabilidade(Reliability);
segurança(Safety) e proteção (Security). Além disso, a visualização desses aspectos
possibilitou um entendimento ainda maior da Buzzmap e dos
componentes de sua estrutura.

Outro aspecto importante, proporcionado por esse trabalho, foi o conhecimento
construído em relação a dependência dos componentes do sistema e os impactos
que cada um deles possui na ferramenta como um todo. Esse conhecimento foi
adquirido por meio da análise de sensibilidade realizada, que possibilitou
observar a influência isolada desses componentes. 

A maior dificuldade de todo projeto, entretanto, esteve sempre relacionada à
cláusula de privacidade presente no contrato da ferramenta Buzzmap, que
impossibilitou uma exibição de dados mais sensíveis da mesma. Desse modo, ao
contrário do que ocorreu com outros grupos, apesar de possuírmos uma boa gama
de informações sobre o sistema, não foi possível, por exemplo, exibir aspectos
como o diagrama de classes completo da ferramenta.

Enfim, o trabalho proporcionou diversos pontos positivos para todos os membros
do grupo, pois contribuiu para um maior conhecimento da ferramenta e abriu
nossas mentes para futuras melhorias e novos caminhos.
