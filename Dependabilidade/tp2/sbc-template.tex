\documentclass[12pt]{article}

\usepackage{sbc-template}

\usepackage{graphicx,url}

%\usepackage[brazil]{babel}   
\usepackage[utf8]{inputenc}  

     
\sloppy

\title{Análise de Dependabilidade da BuzzMap: Ferramenta para Inteligência
Competitiva em Redes Sociais}

\author{Daniel Galinkin\inst{1}, Luiz Mendes\inst{1}, Victor Salles\inst{1}}


\address{
Instituto de Ciências Exatas -- Universidade Federal de Minas Gerais 
  (UFMG)\\
  Caixa Postal 702 -- 31.126-970 -- Belo Horizonte -- MG -- Brazil\\
  Departamento de Ciência da Computação\\
  \email{\{dggc,lfmendes,josue\}@dcc.ufmg.br}
}

\begin{document} 

\maketitle

\begin{abstract}
  This meta-paper describes the style to be used in articles and short papers
  for SBC conferences. For papers in English, you should add just an abstract
  while for the papers in Portuguese, we also ask for an abstract in
  Portuguese (``resumo''). In both cases, abstracts should not have more than
  10 lines and must be in the first page of the paper.
\end{abstract}
     
\begin{resumo} 
  Este meta-artigo descreve o estilo a ser usado na confecção de artigos e
  resumos de artigos para publicação nos anais das conferências organizadas
  pela SBC. É solicitada a escrita de resumo e abstract apenas para os artigos
  escritos em português. Artigos em inglês deverão apresentar apenas abstract.
  Nos dois casos, o autor deve tomar cuidado para que o resumo (e o abstract)
  não ultrapassem 10 linhas cada, sendo que ambos devem estar na primeira
  página do artigo.
\end{resumo}

\section{Descrição do sistema}
Buzzmap é uma ferramenta que propicia o serviço de inteligência
competitiva em redes sociais, criada com o intuito de facilitar e
auxiliar o monitoramento de marcas e opiniões na internet, vigiando
canais de comunicação como Twitter e Youtube.

Esta ferramenta é utilizada basicamente por dois tipos de usuários:
\begin{description}
    \item[Analista] Este usuário da ferramenta a utiliza com o intuito
    de analisar o conteúdo coletado dos canais de comunicação, e gerar
    relatórios e gráficos que sintetizam informações valiosas a
    respeito de um certo assunto.

    \item[Cliente] Este usuário da ferramenta a utiliza com o intuito
    de visualizar os relatórios gerados pelos analistas a respeito de
    determinado assunto.
\end{description}

O sistema está em desenvolvimento e em uso há aproxidamente 10 meses,
sendo que o uso provê feedback e sugestões para a equipe
desenvolvimento.

\section{Definição dos requisitos, atores e casos de uso}

\subsection{Atores}
\begin{description}
    \item[Desenvolvedor] Desenvolvedores do sistema em si. Esse
    projeto esta em constante contrução e aperfeiçoamento, logo o
    desenvolvedor.

    \item[Administrador] É um analista com maiores permissões. Além de
    analisar, administra o projeto e os analistas.

    \item[Analista] Usuário principal do sistema, faz análises de
    conteúdo buscado na internet.

    \item[Cliente] Empresa ou pessoa que contratou o serviço
    oferecido.
\end{description}


\subsection{Requisitos}
\begin{itemize}
    \item A interface gráfica tem que ser acessível via um browser
    pela internet.
    \item Dado um conjunto de palavras chaves o sistema tem de ter a
    capacidade de coletar conteúdo em diversos canais(orkut, flickr,
            blog, twitter).
    \item O sistema deve ter uma interface gráfica que mostra o
    conteudo coletado.
    \item O sistema deve ter uma interface que dá capacidade de se
    analisar o conteúdo coletado.
    \item O sistema deve ser capaz de gerar gráficos sobre a análise
    feita.
\end{itemize}

\subsection{Requisitos de Dependabilidade}
\begin{description}
    \item[Disponibilidade do servidor] O sistema tem que ficar no ar
    durante pelo menos 90\% do tempo durante os horários de análise -
    8:00 as 20:00.
    \item[Safety] O sistema deve fazer um backup do banco de dados de
    12 em 12 horas.
    \item[Security] O sistema deve manter dados de usuários(login e
            senhas) protegidos.
    \item[Security] O sistema deve ter uma divisão de privilégios,
    sendo que algumas informações só podem ser vistas por certos
    usuários.
\end{description}
 


\section{Casos de Uso}

\begin{itemize}
    \item Usuário $->$ loga no sistema

Todo usuário, deve poder logar na ferramenta através de seu usuário e senha.



    \item Analista $->$ Loga no sistema $->$ Analise um projeto

O analista, entra no sistema e após escolher o projeto e o
canal(orkut, youtube...) faz a análise  de conteúdo e salva essa
análise no banco de dados.


\item Analista $->$ Loga no Sistema $->$ Visualiza Gráficos

O analista loga no sistema, escolhe um projeto e visualiza os gráficos do conteúdo.



\item Administrador $->$ Loga no Sistema $->$ Cria um projeto 

Administrador loga no sistema, e cria um projeto, com data inicial e final e palavras chaves.


\item Administrador $->$ Loga no Sistema $->$ Adiciona analista

Administrador loga no sistema, adiciona um analista e o associa a um projeto.



\item Adminstrador $->$ Loga no Sistema $->$ Cria relatório

Administrador loga no sistema, entra num projeto e cria um relatório e envia ao cliente.



\item Cliente $->$ Loga no sistema $->$ Visualiza informações provida pelo administrador sobre um determinado projeto

Cliente loga no sistema, visualiza relatório provido pelo administrador

\end{itemize}

\chapter{Casos de uso}

\begin{figure}[hb]
    \begin{center}
        \includegraphics[scale=0.5]{img/caso-uso-administrador.png}
        \caption{Casos de uso do administrador}
        \label{fig:caso-uso-administrador}
    \end{center}
\end{figure}

\begin{figure}[hb]
    \begin{center}
        \includegraphics[scale=0.5]{img/caso-uso-analista.png}
        \caption{Casos de uso do analista}
        \label{fig:caso-uso-analista}
    \end{center}
\end{figure}

\begin{figure}[hb]
    \begin{center}
        \includegraphics[scale=0.5]{img/caso-uso-cliente.png}
        \caption{Casos de uso do cliente}
        \label{fig:caso-uso-cliente}
    \end{center}
\end{figure}

\chapter{Especificação UML do Diagrama de Classes do Sistema}

Essa parte do trabalho, que deveria especificar os atributos de cada classe,
assim como as associações entre as classes não pode ser realizada, pois a
ferramenta está atrelada a uma cláusula de sigilo de informações. Por isso,
não podemos divulgar a estrutura das classes com a qual a BuzzMap foi
desenvolvida.


\section{Diagramas de atividades}

\begin{figure}[hb]
    \begin{center}
        \includegraphics[scale=0.5]{img/atividade1.png}
        \caption{Diagrama de atividades para o caso de uso em que o
            cliente visualiza um relatório.}
        \label{fig:atividade-visualiza}
    \end{center}
\end{figure}

\begin{figure}[hb]
    \begin{center}
        \includegraphics[scale=0.5]{img/atividade2.png}
        \caption{Diagrama de atividades para o caso de uso em que um
            analista gera um relatório.}
        \label{fig:atividade-analisa}
    \end{center}
\end{figure}

\begin{figure}[hb]
    \begin{center}
        \includegraphics[scale=0.5]{img/atividade3.png}
        \caption{Diagrama de atividades para o caso de uso em que um
            administrador cria um projeto.}
        \label{fig:atividade-cria}
    \end{center}
\end{figure}

\include{questao6}
\chapter{Visão Geral da Arquitetura e Componentes}


\section{Fase 01}

\begin{figure}[hb]
    \begin{center}
        \includegraphics[scale=0.5]{img/conceitual}
        \caption{Arquitetura conceitual geral do sistema}
        \label{fig:arquitetura-conceitual}
    \end{center}
\end{figure}

\section{Fase 02}

\begin{figure}[hb]
    \begin{center}
        \includegraphics[scale=0.5]{img/interface}
        \caption{Arquitetura dos subsistemas da interface}
        \label{fig:arquitetura-interface}
    \end{center}
\end{figure}

\begin{figure}[hb]
    \begin{center}
        \includegraphics[scale=0.5]{img/coletor}
        \caption{Arquitetura dos subsistemas do coletor}
        \label{fig:arquitetura-coletor}
    \end{center}
\end{figure}

\section{Fase 03}

\begin{figure}[hb]
    \begin{center}
        \includegraphics[scale=0.5]{img/jsp}
        \caption{Visão de camadas da implantação JSP}
        \label{fig:}
    \end{center}
\end{figure}


\chapter{Conclusão}
Com isso conclui-se esta fase do projeto, em que foram extraídas as
propriedades de dependabilidade do sistema, como confiabilidade,
segurança, disponibilidade e integridade.

A principal dificuldade encontrada foi a cláusula de privacidade no
contrato da ferramenta Buzzmap, que impossibilitou a exibição de dados
mais sensíveis da mesma, como seu diagrama de classes completo. No
entanto, acredita-se que foi possível realizar um estudo da mesma
tendo em vista o foco deste trabalho, que são suas propriedades de
dependabilidade.



\nocite{*}
\bibliographystyle{sbc}
\bibliography{sbc-template}

\end{document}
