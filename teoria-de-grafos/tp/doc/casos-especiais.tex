\section{Casos especiais}
\label{sec:casos-especiais}

\begin{itemize}
    \item O conjunto de vértices de um grafo é uma cobertura de
    vértices, ainda que a maior possível.

    \item No caso de um grafo $G = (V, E=\empty)$, a cobertura de vértices
    mínima $C = V$, uma vez que os vértices estão isolados.

    \item Em um grafo estrela, a cobertura mínima de vértices é o nó
    central.

    \item Em um grafo euleriano $G = (V, E)$,  existe uma cobertura de
    vértices $C$ tal que $\left| C \right| = \left \lceil {\left| V \right|
        \over 3} \right \rceil$. Esta cobertura é mínima se o grafo só
        contiver as arestas do ciclo euleriano.

    \item Em qualquer grafo $G = (V, E) \mid \exists v \in V, d(v) =
    \left| V \right| - 1$, a cobertura mínima de vértices é composta
    por apenas um vértice que atenda a esta condição.
\end{itemize}
