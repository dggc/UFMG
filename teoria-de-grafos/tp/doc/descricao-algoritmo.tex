\section{Descrição do algoritmo e prova de correção}
\label{sec:descricao-algoritmo}
%TODO: dar uma relida

O algoritmo funciona de acordo com o seguinte pseudo-algoritmo:

HEURÍSTICA($G = (V, E)$)
\begin{enumerate}
    \item Seja $H = (V', E')$ uma cópia do grafo $G= (V, E)$.
    \item Seja $CMV = \emptyset$ o conjunto de vértices que forma a
    cobertura mínima de vértices.
    \item Enquanto $E' \ne \emptyset$, faça:
    \label{item:loop}

    \begin{enumerate}
        \item Caso exista $v \in V' \mid d[v] = 1$:
        \label{item:alone}

        \begin{enumerate}
            \item Seja $u$ o único vértice adjacente a $v$ em $V'$.
            \label{item:find}

            \item Faça $chosen = u$.
        \end{enumerate}
        \item Caso contrário:
        \begin{enumerate}
            \item Seja $u \in V' \mid d[u] \ge d[v], \forall v \in
            V'$.
            \label{item:max}

            \item Faça $chosen = u$.
        \end{enumerate}
        \item Remova de $E'$ toda aresta $e$ incidente a $chosen$.
        \label{item:remove}

        \item Faça $CMV = CMV \cup \{chosen\}$.
    \end{enumerate}
    \item Retorne $CMV$.
\end{enumerate}

Para a avaliação de custo, assumira-se que a estrutura de dados
utilizada é uma matriz de adjacência, usando um vetor auxiliar para
manter e atualizar os graus de incidência de cada vértice a custo
$O(1)$.

O custo para se avaliar a linha~\ref{item:alone} é de $O(V)$
operações, pois é necessário visitar cada nodo para buscar algum que
tenha apenas uma aresta incidente.

O custo para se avaliar a linha~\ref{item:find} é de $O(V)$, pois é
preciso percorrer todos os vértices para achar aquele que é adjacente
ao vértice escolhido.

O custo para se avaliar a linha~\ref{item:max} é de $O(V)$
operações, pois é necessário visitar cada nodo para buscar o que tenha
mais arestas incidentes.

No entanto, é importante lembrar que apenas uma das
linhas~\ref{item:find} e~\ref{item:max} é executada por ciclo.

O custo para se avaliar a linha~\ref{item:remove} é de $O(V)$
operações no pior caso, pois é preciso passar por todas os vértices do
grafo, conferindo se existe ou não uma aresta entre o nó escolhido e
o nó sendo avaliado, e removê-la caso ela exista.

O laço da linha~\ref{item:loop} é repetido no máximo $\left| E
\right|$ vezes, no caso em que apenas uma aresta é retirada por
iteração.

Portanto, o custo de execução deste algoritmo é de, no pior caso:
$$E(O(V) + O(V) + O(V)) = O(VE)$$
