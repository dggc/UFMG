\documentclass[a4paper,10pt,brazil]{article}
\newcommand{\ultimamodificacao}{20 de junho de 2007}
\pagenumbering{arabic}
\usepackage[utf8]{inputenc}
\usepackage[brazil]{babel}
\usepackage{latexsym}
\usepackage{indentfirst}
\usepackage{epsfig}
\usepackage{graphics}
\usepackage{amsmath}
\usepackage{color}
\usepackage{listings}
\usepackage{verbatim}
\usepackage{dinossauro}
\newcommand{\mc}[3]{\multicolumn{#1}{#2}{#3}}
\setlength{\textwidth}{16cm}
\setlength{\textheight}{23,5cm}
\evensidemargin 0cm
\oddsidemargin 0cm
\topmargin -1cm

\definecolor{gray-soft}{rgb}{0.9,0.9,0.9}
\lstset{backgroundcolor=\color{gray-soft},
        basicstyle=\tiny,
        frame=single,
        language=C++}


\newcommand{\displaypaintbox}[4][1.0]{\scalebox{#1}
{\epsfig{file=#4,width=#2,height=#3,bbllx=0in,bblly=#3,bburx=#2,bbury=0cm,silent=}}}

\title{Teoria de Grafos
\\Trabalho Prático
\\Heurística para o Problema da Cobertura Mínima de Vértices}
\author{Alunos: Bruno Guedes Azevedo Viana
\\Daniel Galinkin da Gama Cerqueira
\\
\\Professor: Sebastián Urrutia
\\
\\Departamento de Ciência da Computa\c{c}ão
\\Universidade Federal de Minas Gerais
\\Belo Horizonte, Minas Gerais, Brasil}


\begin{document}
\maketitle
\thispagestyle{empty}
\newpage
\tableofcontents
\newpage

\part{Descrição do problema}
\label{part:descricao}

Uma cobertura de vértices de um grafo é um subconjunto dos seus
vértices tal que todo vértice do grafo ou pertence a este subconjunto,
ou é adjacente a um vértice que pertence a ele.

A cobertura mínima de vértices (CMV) de um grafo é o conjunto independente
máximo no complemento do mesmo.

%TODO achar aplicacoes para cobertura de vertices e conjunto
%independente

\section{Formulação matemática}
\label{sec:formulacao}

\section{Casos especiais}
\label{sec:casos-especiais}

\section{Complexidade}
\label{sec:complexidade}

\section{Propriedades}
\label{sec:propriedades}

\section{Relação com outros problemas}
\label{sec:relacao}

\begin{itemize}
    \item A CMV em um grafo é o conjunto independente
    máximo no grafo complementar~\cite{cite:wikimvc}.

    \item O problema da CMV pode ser formulado como um problema de programação
    linear, cujo problema dual é o de encontrar o emparelhamento
    máximo~\cite{cite:wikimvc}.
    %TODO: mais coisas?
\end{itemize}




\part{Implementação}
\label{part:implementacao}

\section{Descrição do algoritmo e prova de correção}
\label{sec:descricao-algoritmo}
%TODO: dar uma relida

O algoritmo funciona de acordo com o seguinte pseudo-algoritmo:

HEURÍSTICA($G = (V, E)$)
\begin{enumerate}
    \item Seja $H = (V', E')$ uma cópia do grafo $G= (V, E)$.
    \item Seja $CMV = \emptyset$ o conjunto de vértices que forma a
    cobertura mínima de vértices.
    \item Enquanto $E' \ne \emptyset$, faça:
    \label{item:loop}

    \begin{enumerate}
        \item Caso exista $v \in V' \mid d[v] = 1$:
        \label{item:alone}

        \begin{enumerate}
            \item Seja $u$ o único vértice adjacente a $v$ em $V'$.
            \label{item:find}

            \item Faça $chosen = u$.
        \end{enumerate}
        \item Caso contrário:
        \begin{enumerate}
            \item Seja $u \in V' \mid d[u] \ge d[v], \forall v \in
            V'$.
            \label{item:max}

            \item Faça $chosen = u$.
        \end{enumerate}
        \item Remova de $E'$ toda aresta $e$ incidente a $chosen$.
        \label{item:remove}

        \item Faça $CMV = CMV \cup \{chosen\}$.
    \end{enumerate}
    \item Retorne $CMV$.
\end{enumerate}

Para a avaliação de custo, assumira-se que a estrutura de dados
utilizada é uma matriz de adjacência, usando um vetor auxiliar para
manter e atualizar os graus de incidência de cada vértice a custo
$O(1)$.

O custo para se avaliar a linha~\ref{item:alone} é de $O(V)$
operações, pois é necessário visitar cada nodo para buscar algum que
tenha apenas uma aresta incidente.

O custo para se avaliar a linha~\ref{item:find} é de $O(V)$, pois é
preciso percorrer todos os vértices para achar aquele que é adjacente
ao vértice escolhido.

O custo para se avaliar a linha~\ref{item:max} é de $O(V)$
operações, pois é necessário visitar cada nodo para buscar o que tenha
mais arestas incidentes.

No entanto, é importante lembrar que apenas uma das
linhas~\ref{item:find} e~\ref{item:max} é executada por ciclo.

O custo para se avaliar a linha~\ref{item:remove} é de $O(V)$
operações no pior caso, pois é preciso passar por todas os vértices do
grafo, conferindo se existe ou não uma aresta entre o nó escolhido e
o nó sendo avaliado, e removê-la caso ela exista.

O laço da linha~\ref{item:loop} é repetido no máximo $\left| E
\right|$ vezes, no caso em que apenas uma aresta é retirada por
iteração.

Portanto, o custo de execução deste algoritmo é de, no pior caso:
$$E(O(V) + O(V) + O(V)) = O(VE)$$

\section{Estruturas de dados}
\label{sec:estruturas}
%TODO: que mais?

Para a representação do grafo, foi utilizada uma matriz de adjacência,
mais um vetor auxiliar para manter o grau de cada vértice do grafo.
Cada vez que uma aresta é incluída na matriz, os valores das posições
dos vértices da aresta neste vetor são incrementadas. Quando uma
aresta é removida, esses valores são decrementados.

\section{Decisões de implementação}
\label{sec:decisoes}
%TODO: que mais tem que falar?



\part{Resultados computacionais}
\label{sec:resultados}
%LOCK Galinkin


\part{Conclusões}
\label{part:conclusoes}
A heurística consegue resultados bons em relação ao ótimo dos grafos
testados, além de o fazer em tempo hábil, apesar da complexidade
$O(VE)$.

Ela é de fácil compreensão e implementação, tornando-a uma candidata
interessante para resolver o problema da cobertura de vértices (ou
qualquer um dos problemas duais ou reduzíveis deste problema).

O que este trabalho fica devendo é a estimativa de o quanto pior que o
ótimo fica a resposta dada pela heurística, no pior caso. É muito
difícil realizar este tipo de cálculo, sendo que artigos inteiros são
devotados a esta questão. No entanto, pelo menos para os casos
testados, a resposta gerada ficou dentro de um limite aceitável.



\nocite{*}
\bibliographystyle{amsplain}
\bibliography{tp}

\end{document}
