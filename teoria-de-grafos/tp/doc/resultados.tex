\part{Resultados computacionais}
\label{sec:resultados}
%LOCK Galinkin

A heurística desenvolvida foi aplicada a alguns gráficos clássicos da
literatura.

\section{O tetraedro}
Para o tetraedro~\cite{cite:example-plato} com $n=4$ vértices.

\section{O grafo bipartido de Kuratowski $K_{3,3}$}
Para o grafo bipartido de kuratowski
$K_{3,3}$~\cite{cite:example-kuratowski} com $n=6$ vértices.

\section{O octaedro}
Para o octaedro~\cite{cite:example-plato} com $n=6$ vértices.

\section{O grafo de Bondy-Murphy $G_1$}
Para o grafo de Bondy-Murphy $G_1$~\cite{cite:example-bondy} com $n=7$
vértices.

\section{O grafo roda $W_8$}
Para o grafo roda $W_8$~\cite{cite:example-bondy} com $n=8$ vértices.

\section{O cubo}
Para o grafo cubo~\cite{cite:example-plato} com $n=8$ vértices.

\section{O grafo Petersen}
Para o grafo Petersen~\cite{cite:example-petersen} com $n=10$
vértices.

\section{O grafo de Bondy-Murphy $G_2$}
Para o grafo de Bondy-Murphy $G_2$~\cite{cite:example-bondy} com $n=11$
vértices.

\section{O grafo de Grötzsch}
Para o grafo de Grötzsch~\cite{cite:example-grotzsch} com $n=11$
vértices.

\section{O grafo de Herschel}
Para o grafo de Herschel~\cite{cite:example-herschel} com $n=11$
vértices.

\section{O icosaedro}
Para o icosaedro~\cite{cite:example-plato} com $n=12$ vértices.

\section{O grafo de Bondy-Murphy $G_3$}
Para o grafo de Bondy-Murphy $G_3$~\cite{cite:example-bondy} com $n=14$
vértices.

\section{O grafo de Bondy-Murphy $G_24$}
Para o grafo de Bondy-Murphy $G_4$~\cite{cite:example-bondy} com $n=16$
vértices.

\section{O grafo de Ramsey $R(4,4)$}
Para o grafo de Ramsey $R(4,4)$~\cite{cite:example-ramsey} com $n=17$
vértices.

\section{O grafo de Folkman}
Para o grafo de Folkman~\cite{cite:example-folkman} com $n=20$
vértices.

\section{O dodecaedro}
Para o dodecaedro~\cite{cite:example-plato} com $n=20$ vértices.

\section{O grafo de Tutte-Coxeter}
Para o grafo de Tutte-Coxeter~\cite{cite:example-tutte} com $n=30$
vértices.

\section{O grafo de Thomasen}
Para o grafo de Thomasen~\cite{cite:example-thomasen} com $n=34$
vértices.

\section{O grafo de Berge}
Para o grafo de Berge~\cite{cite:example-berge} com $n=60$ vértices.

\section{O grafo de Witzel}
Para o grafo de Witzel~\cite{cite:example-witzel} com $n=450$
vértices.


