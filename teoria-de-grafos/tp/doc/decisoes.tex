\section{Decisões de implementação}
\label{sec:decisoes}
%LOCK: GUEDES
%TODO: que mais tem que falar?

No melhor caso, o algoritmo sempre avalia a linha~\ref{item:alone}
como verdadeira, i.e., sempre existe um vértice de grau 1 cujo
adjacente pode ser retirado do grafo com sucesso.

Entretanto, na maioria dos casos isto não acontece. Neste caso, a
decisão tomada foi de retirar um dos vértices de maior grau. A razão
para esta decisão é que, retirando este vértice do grafo, retira-se o
maior número de arestas de uma só vez. Como o problema é resolvido
quando o número de arestas do grafo $G'$ é igual a zero, isto
significa que o problema está o mais próximo possível de uma solução,
ainda que esta não seja garantidamente uma escolha que faça parte de
uma solução ótima.
